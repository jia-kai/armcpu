% $File: result.tex
% $Date: Fri Mar 14 21:41:00 2014 +0800
% $Author: jiakai <jia.kai66@gmail.com>

\section{实验成果}

\subsection{展示效果}
除了基本实验要求外,还实现了两个用户态的程序:贪吃蛇和幻灯片放映。运行效果如
\figref{ucore}所示。

其中贪吃蛇还需要产生随机数,熵的来源是CP0中count寄存器的值。
对于幻灯片放映,利用了VGA内存映射区是行优先存储的特性,
实现了流畅的纵向换页的动画效果。
具体而言幻灯片的原始数据以行优先连续存储在flash里,
而静态显示或者播放动画时只需从某一行开始把整个屏幕的内容从
flash拷贝到VGA的内存映射区域即可。
另外,为了在256色的VGA输
出下实现较好的显示效果,对图像预先使用了Floyd-Steinberg
dithering算法进行量化,相关细节请自行查阅文献。

\begin{figure}[!ht]
	\addplot{../ucore.png}
	\caption{\label{fig:ucore}ucore运行效果,左上、右上、左下、右下
	依次为刚进入shell时的画面、执行ls后的输出、贪吃蛇游戏界面、幻灯片播放界面}
\end{figure}


\subsection{开发板测试套件}
本实验的一个附加产品是开发板基本功能的测试套件。

\begin{enumerate}
	\item 整体测试 \\
		对于一个新的开发板,
		可以先尝试在CPLD中烧入\verb|archive/cpld.jed|,
		再在FPGA上烧入\verb|archive/armcpu.bit|。将拨码开关最右一个拨到1,
		表示使用12.5M的时钟频率;其它全部置零。

		在PC上连接好串口线后,\verb|cd utils/memtrans|,
		运行\verb|./run_all.sh|,如果没有出错,
		说明串口、flash和RAM均工作正常。
		随后运行
		\begin{verbatim}
			./controller.py flash write ../../archive/ucore-kernel-initrd
		\end{verbatim}
		烧入ucore的系统镜像。然后接上VGA,插入ps/2键盘,
		运行\verb|../terminal.py|进入终端。
		最后把最左的拨码开关拨到1,表示从flash引导,
		这时应该能看到显示器和串口终端都打印出ucore的引导信息,
		PC的键盘或ps/2键盘均可用于输入。
		此时开发板上的LED如\figref{led}所示,
		其中数码管应显示23,内存LED指示灯快速闪烁,
		数码管旁边的LED最右几个亮起。
		\begin{figure}[!ht]
			\addplot{../led.png}
			\caption{\label{fig:led}ucore运行时的LED情况}
		\end{figure}
	\item LED测试 \\
		烧入\verb|misc_fpga_proj/ledtest|里的工程,
		数码管旁的LED应从右向左扫描。
\end{enumerate}

% vim: filetype=tex foldmethod=marker foldmarker=f{{{,f}}}

