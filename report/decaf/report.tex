% $File: report.tex
% $Date: Fri Jan 31 21:26:14 2014 +0800
% $Author: jiakai <jia.kai66@gmail.com>

\documentclass[a4paper]{article}
\usepackage{amsmath,amssymb,amsthm,fontspec,zhspacing,verbatim,graphicx,minted}
\usepackage[hyperfootnotes=false,colorlinks,linkcolor=black,anchorcolor=black,citecolor=black]{hyperref}
\usepackage[top=2in, bottom=1.5in, left=1in, right=1in]{geometry}

\newcommand{\addplot}[1]{\begin{center}
	\includegraphics[width=0.6\paperwidth]{#1}
\end{center}}

% \textref{marker}{text}
\newcommand{\textref}[2]{\hyperref[#1]{#2}}
\newcommand{\figref}[1]{\hyperref[fig:#1]{Figure~\ref*{fig:#1}}}
\newcommand{\eqnref}[1]{\hyperref[eqn:#1]{(\ref*{eqn:#1})}}
\newcommand{\lemref}[1]{\hyperref[lemma:#1]{Lemma~\ref*{lemma:#1}}}


\zhspacing

\title{编译原理拓展实验报告}
\author{清华大学\\贾开~周昕宇~李铁铮}
\date{\today}

\begin{document}
\maketitle

\tableofcontents

\section{实验目的}
修改decaf java实现,使得它能编译出可在计算机组成原理课上设计的CPU以及ucore操作系
统环境下运行的代码。

\section{实验环境}
\begin{enumerate}
	\item 目标CPU是自行设计的流水CPU,采用简化的MIPS32指令集,
		支持基本算术、流程控制指令,有TLB及异常和中断的处理。
	\item 目标操作系统是ucore,是一个简单的类unix操作系统,
		实现了简单的I/O及进程调度的系统调用,可以在上述CPU上运行。
	\item decaf编译器使用java实现,其完整实现已由老师给出,
		可以把decaf源程序编译成MIPS32汇编代码。
\end{enumerate}


\section{实验方案}
% $File: method.tex
% $Date: Fri Jan 31 21:26:13 2014 +0800
% $Author: jiakai <jia.kai66@gmail.com>

\subsection{汇编指令生成}
由于简化的CPU中并未实现add、sub指令,需要把decaf的MIPS后端里生成add、sub指令的部
分改成addu、subu,区别仅在于溢出时后者不会产生异常。

另外,CPU中也未实现除法指令,不过由于所用测试程序中没有除法运算,因此也未进行
相关修改;如果需要除法,可以用其它指令手动实现除法函数,并把除法翻译成函数调用。

当然,一个更好的方法应该是修改ucore系统,在异常处理中捕捉非法指令异常,并软件模
拟未实现的指令。这样,对于编译器而言,目标机器就是一个标准的MIPS32 CPU了。

\subsection{库函数调用及calling convention}
标准MIPS32使用O32 ABI,函数调用的前四个参数通过\$a0-\$a3四个寄存器传输;
但decaf编译出的程序的参数全都在栈上传递。当然,无论什么calling convention,
只要能自恰,程序本身就应该能正常运行,所以需要解决的问题只有用户程序与
C实现的库函数及操作系统交互的部分。

一种常规解决方案是修改decaf编译器,使得其遵循O32 ABI,直接调用相应的函数。
但这需要对后端进行较大的改动,也会造成与现有decaf编译出的二进制代码的不兼容。

在这里,如果把我们的MIPS系统看作一个要移植到的目标平台,
并追求对decaf尽量少的改动,可以采用一种逆向的思路:在decaf和库函数之间增加一个
适配器层,将decaf的调用约定翻译成O32 ABI再调用库函数。
我们的实验中采取了这种方案,用汇编实现了这样的中间层,转发对库函数的调用。

\subsection{程序入口及退出}
我们直接使用了ucore里的linker script(\verb|user.ld|)以及用户静态函数库
\verb|libuser.a|,在该环境下系统会设置好一些全局变量,然后跳转到main执行。
我们修改了decaf编译器,将其输出的\verb|main|重命名为\verb|decaf_main|,
然后汇编实现了一个新的main函数。由于decaf的main是void类型,
我们便默认其都执行成功返回0,于是在\verb|decaf_main|返回后直接调用exit(0)。

% vim: filetype=tex foldmethod=marker foldmarker=f{{{,f}}}



\section{实现细节}
该项目所有代码都在\url{https://git.net9.org/armcpu-devteam/armcpu/tree/master}
上,与decaf相关的部分是/decaf及/ucore/ours/decaf。

\subsection{对decaf编译器的修改}
对decaf编译器的修改以diff形式显示如下,注意其中将add、sub改成了addu、subu,
以及\verb|main|被重命名为\verb|decaf_main|。
\inputminted[linenos,fontsize=\small,tabsize=4]{diff}{../decaf.diff}


\subsection{库函数调用转发及程序入口}
我们实现了以下的汇编程序,用于对库函数调用的转发,
同时注意其中main是程序的入口点。
\inputminted[linenos,fontsize=\small,tabsize=4]{nasm}{../entry.s}

\subsection{库调用的实现}
decaf中有Alloc, ReadInteger, PrintString等函数调用作为语言的内建函数,
可以以保留字的形式直接调用。对此,我们在c中实现了用到的几个函数,具体如下:
\inputminted[linenos,fontsize=\small,tabsize=4]{c}{../userlib.c}


\section{实验效果}
我们成功编译并运行了以下程序,它读入一个整数$N$,
输出$N$个盘子的汉诺塔移动方案。该程序说明了库函数调用、程序入口及退出以及基于
decaf本身calling convention的递归调用的实现的正确性。
\inputminted[linenos,fontsize=\small,tabsize=4]{java}{../hanoi.decaf}

\end{document}

% vim: filetype=tex foldmethod=marker foldmarker=f{{{,f}}}

