% $File: method.tex
% $Date: Fri Jan 31 21:26:13 2014 +0800
% $Author: jiakai <jia.kai66@gmail.com>

\subsection{汇编指令生成}
由于简化的CPU中并未实现add、sub指令,需要把decaf的MIPS后端里生成add、sub指令的部
分改成addu、subu,区别仅在于溢出时后者不会产生异常。

另外,CPU中也未实现除法指令,不过由于所用测试程序中没有除法运算,因此也未进行
相关修改;如果需要除法,可以用其它指令手动实现除法函数,并把除法翻译成函数调用。

当然,一个更好的方法应该是修改ucore系统,在异常处理中捕捉非法指令异常,并软件模
拟未实现的指令。这样,对于编译器而言,目标机器就是一个标准的MIPS32 CPU了。

\subsection{库函数调用及calling convention}
标准MIPS32使用O32 ABI,函数调用的前四个参数通过\$a0-\$a3四个寄存器传输;
但decaf编译出的程序的参数全都在栈上传递。当然,无论什么calling convention,
只要能自恰,程序本身就应该能正常运行,所以需要解决的问题只有用户程序与
C实现的库函数及操作系统交互的部分。

一种常规解决方案是修改decaf编译器,使得其遵循O32 ABI,直接调用相应的函数。
但这需要对后端进行较大的改动,也会造成与现有decaf编译出的二进制代码的不兼容。

在这里,如果把我们的MIPS系统看作一个要移植到的目标平台,
并追求对decaf尽量少的改动,可以采用一种逆向的思路:在decaf和库函数之间增加一个
适配器层,将decaf的调用约定翻译成O32 ABI再调用库函数。
我们的实验中采取了这种方案,用汇编实现了这样的中间层,转发对库函数的调用。

\subsection{程序入口及退出}
我们直接使用了ucore里的linker script(\verb|user.ld|)以及用户静态函数库
\verb|libuser.a|,在该环境下系统会设置好一些全局变量,然后跳转到main执行。
我们修改了decaf编译器,将其输出的\verb|main|重命名为\verb|decaf_main|,
然后汇编实现了一个新的main函数。由于decaf的main是void类型,
我们便默认其都执行成功返回0,于是在\verb|decaf_main|返回后直接调用exit(0)。

% vim: filetype=tex foldmethod=marker foldmarker=f{{{,f}}}
